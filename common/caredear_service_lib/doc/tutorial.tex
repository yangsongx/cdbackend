\documentclass[a4paper]{article}
\title{\xkai{服务器并发处理框架文档}}
\author{CareDear Service Team}
\usepackage{fontspec,xunicode,xltxtra,makeidx,xecolor}
\usepackage{color,listings,tabularx,amsfonts}
\usepackage{amssymb}
\usepackage{titlesec}
\usepackage{epigraph}
\usepackage{soul}
\usepackage[bookmarks=true,pdfborder={0 0 0}]{hyperref}
\setmainfont{WenQuanYi Micro Hei}
\setsansfont{WenQuanYi Micro Hei}
\setmonofont{WenQuanYi Micro Hei}

\newcommand\vtextvisiblespace[1][.4em]{%
  \mbox{\kern.06em\vrule height.5ex}%
  \vbox{\hrule width#1}%
  \hbox{\vrule height.5ex}}

\setlength{\parindent}{0pt}
\setlength{\parskip}{0.5\baselineskip}
\XeTeXlinebreaklocale "zh"
\XeTeXlinebreakskip=0pt plus 1pt minus 0.1pt
\newcommand\fontnamekai{KaiTi}
\newfontinstance\KAI {\fontnamekai}
\newcommand{\kai}[1]{{\KAI#1}}
\newcommand\fontnamexkai{STXingkai}
\newfontinstance\XKAI {\fontnamexkai}
\newcommand{\xkai}[1]{{\XKAI#1}}
\newcommand\fontnamelisu{LiSu}
\newfontinstance\LISU {\fontnamelisu}
\newcommand{\lisu}[1]{{\LISU#1}}
%listing global settings
\lstset{basicstyle=\scriptsize,frame=lines}

%some global color
\definecolor{light-gray}{rgb}{0.87,0.87,0.87}
\definecolor{light-yellow}{rgb}{0.88,0.92,0.48}
\definecolor{mygreen}{rgb}{0.63,1,0.35}

%complicated def
\lstnewenvironment{myjavacode}[1][]
      {\lstset{language=Java}\lstset{escapeinside={(*@}{@*)},
       basicstyle=\footnotesize\ttfamily,
       numbers=left,numberstyle=\scriptsize,stepnumber=1,numbersep=5pt,
       breaklines=true,
       %firstnumber=last,
           %frame=tblr,
           framesep=5pt,
           showstringspaces=false,
           keywordstyle=\itshape\color{blue},
          %identifierstyle=\ttfamily,
           stringstyle=\xecolor{maroon},
        commentstyle=\color{black},
        rulecolor=\color{black},
        xleftmargin=0pt,
        xrightmargin=0pt,
        aboveskip=\medskipamount,
        belowskip=\medskipamount,
               backgroundcolor=\color{white}, #1
}}
{}

%\titleformat{\section}[block]{\kai}{\thesection}{10pt}{}
\makeindex
\begin{document}

\maketitle

\tableofcontents

\begin{abstract}
\kai{本文档介绍一种处理并发的服务器后台运行程序框架,暂叫成\colorbox{mygreen}{CDS(CareDear Service)}}
\end{abstract}

\section{简介}
目前服务器程序框架取自memcached(\cite{refMemcach}),并计划将当中数据库处理逻辑简化并替换成自己需要的内容。

  \subsection{依赖关系}
代码基于如下软件版本开发:

\bgroup
\def\arraystretch{1.15}
\begin{tabular}{|l|l|}
\hline
{依赖包} & {说明}\\
\hline
{libevents 2.0.21-stable} & {event通知库}\\
{memcached 1.4.20} & {程序框架开发基于版本}\\
\hline
\end{tabular}
\egroup

同时还需要如下软件包/库:
\bgroup
\def\arraystretch{1.15}
\begin{tabular}{|l|l|}
\hline
{依赖包} & {说明}\\
\hline
{MySQL-dev} & {C语言访问MySQL的函数借口}\\
{LibXML2} & {解析XML配置文件}\\
{LibCrypto} & {用来解密token key}\\
\hline
\end{tabular}
\egroup

  \subsection{编译代码}
当前程序库支持binary/.so/.a三种不同编译结果。为简化执行程序部署步骤,默认情况下程序是静态库编译,所以文件比较大。

编译方法:

\begin{myjavacode}[numbers=none]
$ make cds       (*@ \colorbox{mygreen}{可运行二进制文件cds}@*)
$ make cs        (*@ \colorbox{mygreen}{动态库文件libcds.so}@*)
$ make ca        (*@ \colorbox{mygreen}{静态库文件cds.a}@*)
$ make testcode  (*@ \colorbox{mygreen}{编译测试demo代码,倚赖.so库}@*)
$ make all       (*@ \colorbox{mygreen}{默认编译结果,生成上述所有文件}@*)
\end{myjavacode}

若想编译成动态库方式,则在编译时指定STATICLIB数值为false:
\begin{myjavacode}[numbers=none]
$ make STATICLIB=false              (*@ \colorbox{mygreen}{动态链接可执行程序}@*)
$ make STATICLIB=false DEBUG=false  (*@ \colorbox{mygreen}{动态链接可执行程序,且是release版本}@*)
\end{myjavacode}

    \subsection{运行服务器状态查看}
从21k-code机器上使用alias的ksxmpp登录,然后在登录上的机器上再使用ksapp3来登录运行程序log存放的位置/home/tokenverify
    \subsection{最简单服务端调用}
框架代码初步设计思想是负责并发请求的接受以及数据的读写。数据处理逻辑由各个调用者来完成。

test/server\_demo.c演示如下场景:
\begin{itemize}
\item Client发送数值请求
\item 服务端利用框架接收和返回处理结果。
\item 服务端代码主要是处理Client发来的数据,demo代码是简单double.
  \end{itemize}
代码:
\begin{myjavacode}
(*@\colorbox{mygreen}{服务端自己提供处理数据的回调函数,框架只管数据获取和返回}@*)
int (*@\hl{my\_handler}@*)(int size, void *req, int *len, void *resp){
    printf("i will process %s\n", param);
    int i = atoi(param);
    if(i  > 0)
    {   
        i = (i * 2); 
        sprintf(resp, "H->%d", i);
        *len = strlen(resp);
    }   
    return 0;
}

int main(int argc, char **argv){
    struct addition_config cfg; 
    cfg.ac_cfgfile = "config.xml";
    cfg.ac_handler = my_handler;
    cds_init((*@\hl{\&cfg, argc, argv}@*));

    return 0;
}
\end{myjavacode}

\section{Token验证模块}
  \subsection{简介}
该模块代码置于token\_auth\_service/目录下。主要功能是校验token的合法性,及更新数据库中的用户token过期时间信息。

  \subsection{消息交互}
    
\section{家庭圈服务模块}
  \subsection{简介}
代码位置位于circle\_mgr\_service/目录

  \subsection{消息交互}
  
\section{网盘服务模块}
    
\section{使用范例}
  \subsection{配置文件}
程序运行中的参数配置目前定义在/etc/cds\_cfg.xml文件中。

xmpp的校验程序(auth)以及token有效期检查等程序都公用这个配置文件,其配置信息如下:
\begin{myjavacode}[numbers=none]
<config>
  (*@ \colorbox{mygreen}{<tokenserver>}@*)
    <ip>127.0.0.1</ip>
    <port>11234</port>
  (*@ \colorbox{mygreen}{</tokenserver>}@*)
  
  (*@ \colorbox{mygreen}{<sqlserver>}@*)
    <ip>10.128.0.27</ip>
    <port>0</port>
    <user>ucen</user>
    <password>heqi</password>
    <databasename>ucen</databasename>
  (*@ \colorbox{mygreen}{</sqlserver>}@*)
</config>
\end{myjavacode}

这些配置信息用于控制程序运行时使用的参数:
\begin{itemize}
\item \colorbox{mygreen}{tokenserver}段是配置auth程序需要连接的token服务端IP和端口值
\item \colorbox{mygreen}{sqlserver}段是设置token服务器需要访问的MySQL机器的IP地址,端口,用户名,密码以及数据库名称。
\end{itemize}

在实际使用过程中,我们可以修改配置文件来调整程序运行参数,达到不用重新编译程序的目的。

  
  \subsection{Token验证}
  \subsection{处理错误码}
服务端的处理结果使用int(4-byte)通知客户端。

其错误码定义:

\bgroup
\def\arraystretch{1.15}
\begin{tabular}{|l|l|}
\hline
{错误码代号} & {说明}\\
\hline
{CDS\_OK(0)} & {处理正确的结果}\\
{CDS\_ERR\_REQ\_TOOLONG(1)} & {客户端请求数据超出范围}\\
{CDS\_ERR\_NOMEMORY(2)} & {内存不足}\\
{CDS\_ERR\_REQ\_INVALID(3)} & {客户端请求的数据格式无效}\\
{CDS\_ERR\_UNMATCH\_USER\_INFO(4)} & {客户端请求的用户信息和token的不一致}\\
{CDS\_ERR\_USER\_TOKEN\_EXPIRED(5)} & {用户的token已经过有效期}\\
{CDS\_ERR\_SQL\_DISCONNECTED(6)} & {服务器端的MySQL访问出错}\\
{CDS\_ERR\_NO\_RESOURCE(7)} & {系统无可用资源(如无法正确生成IPC对象等)}\\
\hline
\end{tabular}
\egroup

定义的代码文件是cds\_public.h

  \subsection{客户端示例}
客户端代码简单地通过TCP连接server相应端口。

示例代码test/client\_demo.c

\section{数据库操作}



\section{TODO List}
\begin{itemize}
\item 清除无用代码
\item 去掉所有编译Warning
\item 增加及优化库函数的原型定义
  \end{itemize}

\begin{thebibliography}{99}
\bibitem{refMemcach}memcached: {\em http://memcached.org}, a high-performance cache system.
\bibitem{reflibevent} Libevent: {\em http://libevent.org}, an open source event notification library.
\end{thebibliography}

\end{document}
