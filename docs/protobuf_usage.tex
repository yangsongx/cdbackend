\documentclass[a4paper]{article}
\title{\kai{protobuf使用}}
\author{CareDear Service Team}
\usepackage{fontspec,xunicode,xltxtra,makeidx,xecolor}
\usepackage{color,listings,tabularx,amsfonts}
\usepackage{amssymb}
\usepackage{titlesec}
\usepackage{epigraph}
\usepackage{soul}
\usepackage[bookmarks=true,pdfborder={0 0 0}]{hyperref}
\setmainfont{WenQuanYi Micro Hei}
\setsansfont{WenQuanYi Micro Hei}
\setmonofont{WenQuanYi Micro Hei}

\newcommand\vtextvisiblespace[1][.4em]{%
  \mbox{\kern.06em\vrule height.5ex}%
  \vbox{\hrule width#1}%
  \hbox{\vrule height.5ex}}

\setlength{\parindent}{0pt}
\setlength{\parskip}{0.5\baselineskip}
\XeTeXlinebreaklocale "zh"
\XeTeXlinebreakskip=0pt plus 1pt minus 0.1pt
\newcommand\fontnamekai{KaiTi}
\newfontinstance\KAI {\fontnamekai}
\newcommand{\kai}[1]{{\KAI#1}}
\newcommand\fontnamexkai{STXingkai}
\newfontinstance\XKAI {\fontnamexkai}
\newcommand{\xkai}[1]{{\XKAI#1}}
\newcommand\fontnamelisu{LiSu}
\newfontinstance\LISU {\fontnamelisu}
\newcommand{\lisu}[1]{{\LISU#1}}
%listing global settings
\lstset{basicstyle=\scriptsize,frame=lines}

%some global color
\definecolor{light-gray}{rgb}{0.87,0.87,0.87}
\definecolor{light-yellow}{rgb}{0.88,0.92,0.48}
\definecolor{mygreen}{rgb}{0.63,1,0.35}

%complicated def
\lstnewenvironment{myjavacode}[1][]
      {\lstset{language=Java}\lstset{escapeinside={(*@}{@*)},
       basicstyle=\footnotesize\ttfamily,
       numbers=left,numberstyle=\scriptsize,stepnumber=1,numbersep=5pt,
       breaklines=true,
       %firstnumber=last,
           %frame=tblr,
           framesep=5pt,
           showstringspaces=false,
           keywordstyle=\itshape\color{blue},
          %identifierstyle=\ttfamily,
           stringstyle=\xecolor{maroon},
        commentstyle=\color{black},
        rulecolor=\color{black},
        xleftmargin=0pt,
        xrightmargin=0pt,
        aboveskip=\medskipamount,
        belowskip=\medskipamount,
               backgroundcolor=\color{white}, #1
}}
{}

%\titleformat{\section}[block]{\kai}{\thesection}{10pt}{}
\makeindex
\begin{document}

\maketitle

\tableofcontents

\begin{abstract}
\kai{本文档介绍Google的服务器间RPC机制-protobuf}
\end{abstract}

\section{简介}
这里暂时以Google的protobuf-2.5.0版本为基准,protobuf的完整代码包,可以在Caredear服务器的backend/3rd\_party/protobuf-2.5.0/目录下找到。

代码编译出的结果是一个编译解析器(protoc),目前支持下面三种语言的绑定:

\begin{enumerate}
 \item C++
 \item Python
 \item Java
\end{enumerate}

Figure-\ref{figProtoc}是protobuf处理流程,其编译器(protoc)将会把交互消息定义.proto文件(IDL),按照用户指定语言绑定来自动生成相对应代码。
\begin{figure}
\caption{protoc的使用流程}\label{figProtoc}
\centering
\includegraphics[scale=0.80]{protoc.eps}
\end{figure}

各个语言间可以无缝的交互数据,而不必再自己手动地编写代码去解析,或者组装数据包。

\section{Java的绑定}
目前默认下,protobuf只会生成Java代码,运行需要的库,需要使用Apache Maven包(mvn命令行程序)来编译。

在Ubuntu系统上,使用apt-get install mvn既可得到mvn程序,再使用下面命令:

\begin{lstlisting}
protobuf/java $ mvn install
protobuf/java $ mvn package
\end{lstlisting}

这将会在java/目录下生成target/protobuf-java-2.5.0.jar包,Java语言程序编译和运行,都需要依赖该包。

java -classpath protobuf-java-2.5.0.jar:. Main

\section{使用示例}
在caredear服务器代码的backend/docs/example/目录下,是一个演示Java前端和C++后端交换数据的流程。



\section{IDL语法}

\section{关于本文档}
介绍protobuf话题的文档置于backend/docs/目录下。

%% BEGIN CD-Specific CHANGE

%% END CD-Specific CHANGE



\begin{thebibliography}{99}
\bibitem{refMemcach}memcached: {\em http://memcached.org}, a high-performance cache system.
\bibitem{reflibevent} Libevent: {\em http://libevent.org}, an open source event notification library.
\end{thebibliography}

\end{document}

