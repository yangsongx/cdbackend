\documentclass{beamer}
\usetheme{CambridgeUS}
%\usetheme{Hannover}
\usepackage{fontspec,xunicode,xltxtra,xecolor,listings,color}
\usepackage{bchart}
\usepackage{tikz}
\usetikzlibrary{arrows,shapes,trees,calc,automata,positioning}
\tikzset{
  mybox/.style={
    rectangle,
    rounded corners,
    draw=black
  },
}

\setmainfont{WenQuanYi Micro Hei}
\setsansfont{WenQuanYi Micro Hei}
\setmonofont{WenQuanYi Micro Hei}
\XeTeXlinebreaklocale "zh"
\XeTeXlinebreakskip=0pt plus 1pt minus 0.1pt
\newcommand\fontnamekai{KaiTi}
\newfontinstance\KAI {\fontnamekai}
\newcommand{\kai}[1]{{\KAI#1}}
\newcommand\fontnamexkai{STXingkai}
\newfontinstance\XKAI {\fontnamexkai}
\newcommand{\xkai}[1]{{\XKAI#1}}
\newcommand\fontnamelisu{LiSu}
\newfontinstance\LISU {\fontnamelisu}
\newcommand{\lisu}[1]{{\LISU#1}}
\begin{document}
\defverbatim[colored]\apicode{%
  \begin{lstlisting}[basicstyle=\tiny,language=java,frame=lines,showspaces=false,emphstyle=\color{blue}]
import com.caredear.common.util.CdRecycleUsrData;

int CdRecycleUsrData.saveDeletedContactData(String deleted_contact);
int CdRecycleUsrData.restoreDeletedContactData(Context context);
\end{lstlisting}
}
\defverbatim[colored]\mycode{%
  \begin{lstlisting}[basicstyle=\tiny,language=java,frame=lines,showspaces=false,emphstyle=\color{blue}]
import com.caredear.common.util.CdRecycleUsrData;

String deleted_contact="张三:13022593515;李四:02652148888;";
int ret = CdRecycleUsrData.saveDeletedContactData(deleted_contact);
if(ret == 0){
    log("Cool, we did it!");
} else {
    log("error happen for backup the deleted contact!");
}
\end{lstlisting}
}
\defverbatim[colored]\restcode{%
  \begin{lstlisting}[basicstyle=\tiny,language=java,frame=lines,showspaces=false,emphstyle=\color{blue}]
import com.caredear.common.util.CdRecycleUsrData;

private Context mCtx;
mCtx = getApplicationContext()
int ret = CdRecycleUsrData.restoreDeletedContactData(mCtx)
if(ret == 0){
    log("Cool, we did it!");
} else {
    log("error happen for restore the deleted contact!");
}
\end{lstlisting}
}

\title{\xkai{亲连用户数据分析}}
\institute{CearDear Service Team}
\frame{\titlepage}

\begin{frame}
\frametitle{2014年11月份注册用户地理位置分布(合计599)}

\scalebox{0.5}
{
\begin{bchart}[max=120]
\bcbar[text=江苏,color=red]{120}
\bcbar[text=广东,color=orange]{107}
\bcbar[text=北京,color=green!60!blue]{85}
\bcbar[label=上海]{36}
\bcbar[label=浙江]{20}
\bcbar[label=陕西]{20}
\bcbar[label=四川]{20}
\bcbar[label=湖北]{19}
\bcbar[label=河南]{18}
\bcbar[label=山东]{17}
\bcbar[label=天津]{13}
\bcbar[label=河北]{13}
\bcbar[label=湖南]{13}
\bcbar[label=重庆]{13}
\bcbar[label=黑龙江]{12}
\bcbar[label=山西]{11}
\bcbar[label=福建]{10}
\bcbar[label=江西]{8}
\bcbar[label=安徽]{8}
\bcbar[label=辽宁]{8}
\bcbar[label=贵州]{7}
\bcbar[label=广西]{4}
\bcbar[label=新疆]{3}
\bcbar[label=海南]{3}
\bcbar[label=内蒙古]{3}
\bcbar[label=甘肃]{3}
\bcbar[label=云南]{2}
\bcbar[label=吉林]{2}
\bcbar[label=宁夏]{1}
\end{bchart}
}
\end{frame}

\begin{frame}
\frametitle{活跃用户分布情况}

\def\angle{0}
\def\radius{3}
\def\cyclelist{{"red","orange","blue","green"}}
\newcount\cyclecount \cyclecount=-1
\newcount\ind \ind=-1
\scalebox{0.8}
{
\begin{tikzpicture}[nodes = {font=\sffamily}]
  \foreach \percent/\name in {
      67/最近一周登录用户,
      25/一个月内登录用户,
      8/不活跃用户
    } {
      \ifx\percent\empty\else               % If \percent is empty, do nothing
        \global\advance\cyclecount by 1     % Advance cyclecount
        \global\advance\ind by 1            % Advance list index
        \ifnum3<\cyclecount                 % If cyclecount is larger than list
          \global\cyclecount=0              %   reset cyclecount and
          \global\ind=0                     %   reset list index
        \fi
        \pgfmathparse{\cyclelist[\the\ind]} % Get color from cycle list
        \edef\color{\pgfmathresult}         %   and store as \color
        % Draw angle and set labels
        \draw[fill={\color!50},draw={\color}] (0,0) -- (\angle:\radius)
          arc (\angle:\angle+\percent*3.6:\radius) -- cycle;
        \node at (\angle+0.5*\percent*3.6:0.7*\radius) {\percent\,\%};
        \node[pin=\angle+0.5*\percent*3.6:\name]
          at (\angle+0.5*\percent*3.6:\radius) {};
        \pgfmathparse{\angle+\percent*3.6}  % Advance angle
        \xdef\angle{\pgfmathresult}         %   and store in \angle
      \fi
    };
\end{tikzpicture}
}

\end{frame}


\begin{frame}
\frametitle{活跃用户地理位置分布}
\scalebox{0.53}
{
\begin{bchart}
\bcbar[text=广东]{67}
\bcbar[text=江苏]{62}
\bcbar[text=北京]{56}
\bcbar[label=上海]{25}
\bcbar[label=四川]{17}
\bcbar[label=河南]{15}
\bcbar[label=陕西]{15}
\bcbar[label=湖北]{13}
\bcbar[label=浙江]{13}
\bcbar[label=山东]{12}
\bcbar[label=天津]{11}
\bcbar[label=黑龙江]{10}
\bcbar[label=重庆]{9}
\bcbar[label=湖南]{9}
\bcbar[label=河北]{9}
\bcbar[label=山西]{8}
\bcbar[label=辽宁]{7}
\bcbar[label=江西]{7}
\bcbar[label=安徽]{6}
\bcbar[label=贵州]{6}
\bcbar[label=福建]{5}
\bcbar[label=甘肃]{3}
\bcbar[label=内蒙古]{3}
\bcbar[label=广西]{3}
\bcbar[label=新疆]{3}
\bcbar[label=吉林]{2}
\bcbar[label=海南]{2}
\bcbar[label=云南]{2}
\bcbar[label=宁夏]{1}
\end{bchart}
}

\end{frame}


\begin{frame}
\frametitle{注册用户地理位置分布-2014-11-21,358用户}
\scalebox{0.53}
{
\begin{bchart}[max=101]
 \bcbar[text=江苏,color=red]{101}
 \bcbar[text=广东,color=orange]{82}
 \bcbar[text=北京,color=green!60!blue]{39}
 \bcbar[label=上海]{15}
 \bcbar[label=重庆]{11}
 \bcbar[label=河南]{11}
 \bcbar[label=浙江]{10}
 \bcbar[label=陕西]{10}
 \bcbar[label=河北]{9}
 \bcbar[label=山东]{9}
 \bcbar[label=湖南]{7}
 \bcbar[label=湖北]{7}
 \bcbar[label=黑龙江]{7}
 \bcbar[label=天津]{7}
 \bcbar[label=福建]{6}
 \bcbar[label=四川]{4}
 \bcbar[label=贵州]{3}
 \bcbar[label=安徽]{3}
 \bcbar[label=海南]{2}
 \bcbar[label=云南]{2}
 \bcbar[label=辽宁]{3}
 \bcbar[label=江西]{1}
 \bcbar[label=广西]{1}
 \bcbar[label=新疆]{1}
 \bcbar[label=吉林]{1}
 \bcbar[label=宁夏]{1}
\end{bchart}
}

\end{frame}


\begin{frame}
\frametitle{活跃用户分布情况}

\def\angle{0}
\def\radius{3}
\def\cyclelist{{"red","orange","blue","green"}}
\newcount\cyclecount \cyclecount=-1
\newcount\ind \ind=-1
\begin{tikzpicture}[nodes = {font=\sffamily}]
  \foreach \percent/\name in {
      72/最近一周登录用户,
      19/一个月内登录用户,
      9/不活跃用户
    } {
      \ifx\percent\empty\else               % If \percent is empty, do nothing
        \global\advance\cyclecount by 1     % Advance cyclecount
        \global\advance\ind by 1            % Advance list index
        \ifnum3<\cyclecount                 % If cyclecount is larger than list
          \global\cyclecount=0              %   reset cyclecount and
          \global\ind=0                     %   reset list index
        \fi
        \pgfmathparse{\cyclelist[\the\ind]} % Get color from cycle list
        \edef\color{\pgfmathresult}         %   and store as \color
        % Draw angle and set labels
        \draw[fill={\color!50},draw={\color}] (0,0) -- (\angle:\radius)
          arc (\angle:\angle+\percent*3.6:\radius) -- cycle;
        \node at (\angle+0.5*\percent*3.6:0.7*\radius) {\percent\,\%};
        \node[pin=\angle+0.5*\percent*3.6:\name]
          at (\angle+0.5*\percent*3.6:\radius) {};
        \pgfmathparse{\angle+\percent*3.6}  % Advance angle
        \xdef\angle{\pgfmathresult}         %   and store in \angle
      \fi
    };
\end{tikzpicture}


\end{frame}

\begin{frame}
\frametitle{活跃用户地理分布-2014-11-21}
\scalebox{0.53}
{
\begin{bchart}[max=51,color=red]
 \bcbar[text=江苏]{51}
 \bcbar[text=广东]{51}
 \bcbar[text=北京]{37}
 \bcbar[label=上海]{12}
 \bcbar[label=重庆]{10}
 \bcbar[label=河南]{11}
 \bcbar[label=浙江]{10}
 \bcbar[label=河北]{9} 
 \bcbar[label=陕西]{7}
 \bcbar[label=山东]{7}
 \bcbar[label=湖南]{7}
 \bcbar[label=湖北]{6}
 \bcbar[label=黑龙江]{5}
 \bcbar[label=天津]{7}
 \bcbar[label=福建]{5}
 \bcbar[label=四川]{3}
 \bcbar[label=贵州]{3}
 \bcbar[label=安徽]{2}
 \bcbar[label=海南]{2}
 \bcbar[label=云南]{2}
 \bcbar[label=辽宁]{3}
 \bcbar[label=江西]{1}
 \bcbar[label=广西]{1}
 \bcbar[label=新疆]{1}
 \bcbar[label=吉林]{1}
 \bcbar[label=宁夏]{1}
\end{bchart}
}
\end{frame}



\begin{frame}
\frametitle{注册用户地理位置分布: 2014-11-18, 305用户}
\scalebox{0.53}
{
\begin{bchart}[max=92]
 \bcbar[text=江苏,color=red]{92}
 \bcbar[text=广东,color=orange]{77}
 \bcbar[text=北京,color=green!60!blue]{30}
 \bcbar[label=上海]{15}
 \bcbar[label=重庆]{11}
 \bcbar[label=浙江]{9}
 \bcbar[label=陕西]{9}
 \bcbar[label=山东]{8}
 \bcbar[label=湖南]{6}
 \bcbar[label=福建]{6}
 \bcbar[label=河北]{6}
 \bcbar[label=湖北]{5}
 \bcbar[label=河南]{5}
 \bcbar[label=黑龙江]{5}
 \bcbar[label=贵州]{3}
 \bcbar[label=安徽]{3}
 \bcbar[label=四川]{3}
 \bcbar[label=海南]{2}
 \bcbar[label=云南]{2}
 \bcbar[label=辽宁]{2}
 \bcbar[label=江西]{1}
 \bcbar[label=天津]{1}
 \bcbar[label=广西]{1}
 \bcbar[label=新疆]{1}
 \bcbar[label=吉林]{1}
 \bcbar[label=宁夏]{1}
\end{bchart}
}

\end{frame}

\end{document}
