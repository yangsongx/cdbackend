\documentclass{beamer}
\usetheme{Madrid}
%\usetheme{Hannover}
\usepackage{fontspec,xunicode,xltxtra,xecolor,listings,color}
\usepackage{bchart}
\usepackage{tikz}
\usetikzlibrary{arrows,shapes,trees,calc,automata,positioning,decorations}
\tikzset{
  mybox/.style={
    rectangle,
    rounded corners,
    draw=black
  },
}

\definecolor{light-gray}{rgb}{0.87,0.87,0.87}

\setmainfont{WenQuanYi Micro Hei}
\setsansfont{WenQuanYi Micro Hei}
\setmonofont{WenQuanYi Micro Hei}
\XeTeXlinebreaklocale "zh"
\XeTeXlinebreakskip=0pt plus 1pt minus 0.1pt

\begin{document}

\defverbatim[colored]\apicode{%
  \begin{lstlisting}[basicstyle=\tiny,language=java,frame=lines,showspaces=false,emphstyle=\color{blue}]
import com.caredear.common.util.CdRecycleUsrData;

int CdRecycleUsrData.saveDeletedContactData(String deleted_contact);
int CdRecycleUsrData.restoreDeletedContactData(Context context);
\end{lstlisting}
}

\title{亲连服务器开发设计文档}
\institute{CearDear Service Team}
\frame{\titlepage}

\begin{frame}
\frametitle{Outline}
\tableofcontents
\end{frame}


%%%%%%%%%%%%%%%%%%%%%%%%%%%%%%%%%%%%%%%%%%%%%%%%%%%%%%%%%%%%%%
\section{后台服务架构}

%  ######################
\begin{frame}
\frametitle{基础架构}

\begin{tikzpicture}
\node {yz-admin}
  [edge from parent fork down,font=\scriptsize]
  child {node {yz-app-java1}}
  child {node {yz-app-java2}
  child [child anchor=north east]{node {child}}
  child {node {child}} };
\end{tikzpicture}


\end{frame}

%  ######################
\begin{frame}
\frametitle{各个服务分配情况}

\begin{center}

\bgroup
\def\arraystretch{1.15} 
\begin{tabular}{|c|c|}
{服务名称} & {端口号}\\
\hline
{用户注册User Register Service(urs)} & {13000}  \\
{用户登录User Login Service(uls)} & {13001}  \\
{用户激活User Activation Service(acts)} & {13002} \\
{用户鉴权User Authentication(uauth)} & {13003} \\
{用户密码管理Password Manager(passwdmgr)} & {13004} \\
{个人信息修改User Attribute Modify(attrmodify)} & {13005} \\
{用户信息补齐(upusr)} & {13006} \\
{激活码(vcs)} & {13007} \\
\hline
{网盘服务NetDisk Service(nds)} & {12001} \\
{SIPs用户鉴权(opas)} & {12002} \\
\end{tabular}
\egroup

\end{center}


\end{frame}

\begin{frame}
\frametitle{服务的状态码}\label{lblServerStatusCode}

目前服务器对请求的处理结果,使用状态码来标示成功与否,常见定义如下:

\vspace{0.2cm}

\bgroup
\def\arraystretch{1.15} 
\tiny
\begin{tabular}{|c|l|c|l|}
\hline
{状态码} & {说明} & {状态码} & {说明}\\
\hline
{0} & {成功} & {1} & {请求数据过长} \\
{2} & {内存不够} & {3} & {请求数据格式不对} \\
{4} & {用户信息不匹配} & {5} & {token过期} \\
{6} & {数据库连接断开} &  {7} & {SQL语句执行出错}\\
{8} & {数据库中找不到指定记录} & {9} & {系统资源不足} \\
{10} & {文件已存在} & {11} & {Quota超限制} \\
\hline
\end{tabular}
\egroup

\vspace{0.2cm}

代码级别的定义在backend/common/cds\_base/cds\_public.h头文件中。

\end{frame}


\begin{frame}
\frametitle{服务集群}

TODO

\end{frame}


%%%%%%%%%%%%%%%%%%%%%%%%%%%%%%%%%%%%%%%%%%%%%%%%%%%%%%%%%%%%%%
\section{新用户注册设计}

%  ######################
\begin{frame}
\frametitle{申请Token}


\end{frame}

%%%%%%%%%%%%%%%%%%%%%%%%%%%%%%%%%%%%%%%%%%%%%%%%%%%%%%%%%%%%%%
\section{网盘设计}

%  ######################
\begin{frame}[fragile]
\frametitle{上传文件}

\begin{tikzpicture}
% the input from APK
\node [circle,draw,fill=black!20,thick,inner sep=0pt,minimum size=6mm] (IDApk) {APK};
\node [mybox,right of = IDApk,xshift=0.2cm,yshift=2cm] (ID1) {\begin{lstlisting}[frame=none,basicstyle=\tiny]
{
  "user":"130xxxxxxxx",
  "filename":"abc.png",
  "md5":xxxxxxx
}
                     \end{lstlisting}
};

\node [mybox, right of = IDApk, xshift=4cm,font=\small] (IDWeb) {Web前端};

\node [mybox, right of = IDWeb, xshift=4cm,yshift=2cm,font=\small] (IDTauth) {验证服务};
\node [rectangle,above of = IDTauth,yshift=-0.5cm] () {tauth};
\node [mybox, right of = IDWeb, xshift=4cm,yshift=-2cm,font=\small] (IDNetdisk) {网盘服务};

\node [circle,draw=blue!50,fill=blue!20,thick,inner sep=0pt,minimum size=6mm,below of = IDWeb,xshift=-2.5cm, yshift=-1.5cm,font=\small] (IDQiniu) {Qiniu服务器};
%below are arrow lines...
\draw [->, thick] (IDWeb) -- (IDApk);
%below are layer

\end{tikzpicture}
\end{frame}


%  ######################
\begin{frame}
\frametitle{下载文件}

\end{frame}


%  ######################
\begin{frame}
\frametitle{删除文件}

\begin{tikzpicture}
\node [circle,draw=blue!50,fill=blue!20,thick,inner sep=0pt,minimum size=6mm] (ID1) {Good};
\end{tikzpicture}

\end{frame}

%  ######################
\begin{frame}
\frametitle{用户网盘文件列举}

注意事项:当用户文件列表数据过大,多于1K时候,
是否需要通过zip等方式来完成?

\end{frame}

%  ######################
\begin{frame}
\frametitle{网盘数据库}

\begin{tikzpicture}
\node [circle,draw=blue!50,fill=blue!20,thick,inner sep=0pt,minimum size=6mm,font=\scriptsize] (IDdb) {netdisk};

\node [rectangle, below of = IDdb,yshift=-1.5cm] (ID1) {\scriptsize \begin{tabular}{|c|}
\hline
{USERS} \\
\hline
{FILE\_INFO} \\
\hline
{FILES} \\
\hline
                      \end{tabular}};
% tabls..
\node [rectangle, right of = ID1,xshift=5.5cm, yshift=3cm] (IDUsers) {\tiny \begin{tabular}{|l|l|l|}
\hline
{USER\_NAME}  & {USED\_SIZE} & {USER\_QUOTA} \\
\hline
{130xxx} & {32} & {1000}  \\
{138xxx} & {99} & {1000}  \\
\hline
                      \end{tabular}};

\node [rectangle, right of = ID1,xshift=5.5cm] (IDFileinfo) {\tiny \begin{tabular}{|l|l|l|}
\hline
{FILENAME}  & {MD5SUM} & {FILETYPE} \\
\hline
{abc.png}  & {af378123jxx} & {png} \\
\hline
                      \end{tabular}};

\node [rectangle, right of = ID1,xshift=5.5cm, yshift=-3cm] (IDFiles) {\tiny \begin{tabular}{|l|l|l|l|}
\hline
{HASH\_KEY}  & {SIZE} & {FILENAME} & {OWNER}\\
\hline
{FILES}  & {} & {} & {}\\
\hline
                      \end{tabular}};
\end{tikzpicture}

\end{frame}


%  ######################
\begin{frame}
\frametitle{家相册功能设计}
数据库支持:

\begin{itemize}
 \item 添加一个圈属性
\end{itemize}


\begin{tikzpicture}
% tabls..
\node [rectangle] (IDUsers) {\tiny \begin{tabular}{|l|l|l|l|}
\hline
{USER\_NAME}  & {USED\_SIZE} & {USER\_QUOTA} & {\colorbox{blue!30}{CIRCLE}}\\
\hline
{130xxx} & {32} & {1000} & {\colorbox{blue!30}{34}} \\
{138xxx} & {99} & {1000} & {\colorbox{blue!30}{678,900}} \\
\hline
                      \end{tabular}};
\node [rectangle, above of = IDUsers, yshift=-0.15cm, font=\scriptsize] () {USERS};

\node [rectangle, right of = IDUsers, xshift=5cm, yshift=-2cm] (IDcircle) {\tiny \begin{tabular}{|l|l|}
\hline
{circleID}  & {member}\\
\hline
{34} & {130xxx, jerry} \\
{678} & {138xxx, 137xxx} \\
{900} & {138xxx, 177xxx} \\
\hline
                      \end{tabular}};
\node [rectangle, above of = IDcircle, yshift=-0.2cm, font=\scriptsize] {ejarbbed数据库};

\draw ($(IDUsers.east) + (0, 0.0cm)$) -- ($(IDUsers.east) + (1.5cm, 0.0cm)$);
\draw [->] ($(IDUsers.east) + (1.5cm, 0.0cm)$) -- ($(IDUsers.east) + (1.5cm, -1.4cm)$);

\draw ($(IDUsers.south) + (2.4cm, 0.0cm)$) -- ($(IDUsers.south) + (2.4cm, -1.5cm)$);
\draw [->] ($(IDUsers.south) + (2.4cm, -1.5cm)$) -- ($(IDUsers.south) + (4.3cm, -1.5cm)$);
\end{tikzpicture}

\end{frame}


%%%%%%%%%%%%%%%%%%%%%%%%%%%%%%%%%%%%%%%%%%%%%%%%%%%%%%%%%%%%%%
\section{用户中心}
%  ######################
\begin{frame}
\frametitle{数据库}

\begin{tikzpicture}
\node [rectangle](IDtable) {\bgroup
\def\arraystretch{1.15} 
\scriptsize
\begin{tabular}{| l | l | l |}
\hline
{\bf id} & {\bf status} & {height} \\
\hline
{田凤}  & {1994-06-12}   & {162} \\
{戴翊}  & {2002-6-22}     & {170}  \\
{蔡爽爽} & {1995-01-09}    & {165} \\
\hline
\end{tabular}
\egroup
};

\end{tikzpicture}

\end{frame}

%  ######################
\begin{frame}[fragile]
\frametitle{Web请求格式}

Web通过http接受外来请求,每个HTTP请求都需要一个共同的头部内容:

\begin{tikzpicture}

\draw [fill=blue!30](0,0) rectangle (2, 0.6) node [pos=.5,font=\footnotesize] {devicetype};
\draw (2, 0) rectangle (3, 0.6) node [pos=.5,font=\footnotesize] {sysid};
\draw (3, 0) rectangle (5, 0.6) node [pos=.5,font=\footnotesize] {systicket};

\node at (1, 1.5) [rectangle](IDtable) {\bgroup
\def\arraystretch{1.15} 
\tiny
\begin{tabular}{| l | l |}
\hline
{\bf 数值} & {\bf 说明} \\
\hline
{0}  & {Android} \\
{1}  & {iOS}  \\
{2} & {Caredear老人机} \\
{3} & {PC} \\
\hline
\end{tabular}
\egroup
};

\node at (6.5, 1.4) [rectangle](IDtable2) {\bgroup
\def\arraystretch{1.15} 
\tiny
\begin{tabular}{ l | l | l}
\hline
{\bf sysid}  & {systicket} & {\bf 说明} \\
\hline
{1}  & {Android} & {来自官网}\\
{2}  & {638ce788-e3c3-47a7-b994-916b3c6b5efw}  & {app用户中心}\\
{3} & {638ce788-e3c3-47a7-b924-916b3c6b5efd} & {亲连}\\
\hline
\end{tabular}
\egroup
};

\node [above of = IDtable2, yshift=-0.5,font=\scriptsize] {源自数据库uc\_sys\_access表};

\node [mybox, below of = IDtable2, yshift=-0.55, font=\tiny]{\begin{lstlisting}[frame=none]
 curl -H 'sysid:2' -H 'systicket:638ce788-e3c3-47a7-b994-916b3c6b5efw'       
                                              \end{lstlisting}
};
\end{tikzpicture}

\end{frame}

%  ######################
\begin{frame}[fragile]
\frametitle{注册新用户示例}

\begin{tikzpicture}
\node [mybox] () {\begin{lstlisting}[frame=none]
curl -H 'sysid:2' -H 'systicket:638ce788-e3c3-47a7-b994-916b3c6b5efw'
                  \end{lstlisting}
};
\end{tikzpicture}

\end{frame}

%  ######################
\begin{frame}
\frametitle{修改密码}

\begin{tikzpicture}
%\node [mybox] () {\begin{lstlisting}[frame=none]
%curl -H 'sysid:2' -H 'systicket:638ce788-e3c3-47a7-b994-916b3c6b5efw'
%                  \end{lstlisting}
%};
\end{tikzpicture}

\end{frame}

%%%%%%%%%%%%%%%%%%%%%%%%%%%%%%%%%%%%%%%%%%%%%%%%%%%%%%%%%%%%%%
\section{手机定制服务}

%  ######################

\subsection{架构与流程}
\begin{frame}[fragile]
\frametitle{定制服务架构}
基于caredear统一用户中心来制作:

\begin{tikzpicture}
\draw [rounded corners, draw=red!20,fill=red!20, dashed] (2.0, 2.0) rectangle (9.7, -2.5);

\node [mybox, font=\tiny] (IDbrowser) {浏览器};
\node [mybox, below of = IDbrowser,font=\tiny] (IDphone) {21ke手机};

\draw (2.2,-1.5) rectangle (3.5, 0.8) node [pos=.5, font=\tiny] (IDserver) {定制服务器};

\draw (5.0, -1.5) rectangle (6.5, 0.8) node [pos=.5, font=\tiny] (IDc) {C++后台服务};

\node at (8.5, -0.45) [cylinder , aspect=0.25,  draw, minimum width=1.5cm,
                 cylinder body fill=blue!30, 
                 cylinder end fill=blue!20, shape border rotate=90](IDdb){数据库};
\node at (1.3, 0)[double arrow, draw,font=\tiny] {数据交换};
\draw [<->, >=stealth'] (IDphone.east) -- ($(IDphone.east) + (1.4, 0)$) node [pos=.5, above, font=\tiny]() {请求下载};
\draw [<->, >=stealth'] (IDserver.east) -- (IDc.west);
\draw [<->, >=stealth'] (IDc.east) -- ($(IDdb.west) + (0, 0.1cm)$);
\end{tikzpicture}

\end{frame}

%  ######################

\begin{frame}
\frametitle{定制开机动画}
  \framesubtitle{使用场景}
  
\begin{itemize}
 \item 在caredear官网登录用户中心,进入定制相关入口。
 \item 在定制开机动画的子菜单中,选择购卖机型,接着选择本机照片(暂定2张)
 \item 前端根据机型判断出屏幕大小,并在UI上绘制出裁剪区域:\begin{figure}


\includegraphics[scale=0.15]{imgprocdemo.png}
\end{figure}

\end{itemize}

\end{frame}

%  ######################
\begin{frame}
\frametitle{定制开机动画}
  \framesubtitle{使用场景(2)}
  
\begin{itemize}
 \item 根据前端的具体实现\footnote{Qiniu Javascript具有这类插件},提供编辑功能如:
  \begin{enumerate}
    \item 添加文字水印
    \item 背景色调的改变
  \end{enumerate}
  \item 前端生成开机预览,用户确认提交。
  \item 生成一个PIN码类识别符,标示定制过程成功
\end{itemize}


\end{frame}

%  ######################
\begin{frame}[fragile]
\frametitle{定制开机动画}
  \framesubtitle{服务器端处理}

\begin{tikzpicture}
\draw [rounded corners, draw=red, dashed] (0.6, 2.8) rectangle (9.7, -3.0);

\draw [rounded corners, fill=green!30,draw=green!30, dashed] (4.9, 0.3) rectangle (9.3, -0.3);
\draw [rounded corners, fill=red!30,draw=red!30, dashed] (4.9, -0.4) rectangle (9.3, -0.7);

\node[rectangle, draw=black, fill=green!20,font=\scriptsize] (IDok) {确认};

\node [rectangle, right of = IDok, xshift=7.5cm, yshift=1.8cm](IDmaintbl) {\bgroup
\def\arraystretch{1.15} 
\tiny
\begin{tabular}{|l|l|}
\hline
{id} & {user} \\
\hline
{100} & {foo}  \\
\hline
\end{tabular}
\egroup
}; 
\node[rectangle,font=\tiny, above of = IDmaintbl, yshift=-0.5cm] {uc.uc\_passport};

%file table
\node [rectangle, right of = IDok, xshift=6cm](IDtbl) {\bgroup
\def\arraystretch{1.15} 
\scriptsize
\begin{tabular}{|l|l|l|l|}
\hline
{cid} & {md5(key)} & {suffix} & {type} \\
\hline
{100} & {xx-xxxx} & {jpg} & {image} \\
{100} & {yy-yyyy} & {png} & {image} \\
{100} & {zz-zzzz} & {zip} & {boot} \\
\hline
\end{tabular}
\egroup
}; 
\node[rectangle,font=\scriptsize, above of = IDtbl, yshift=-0.02cm] {uc.makerfiles};

%server process steps
\node[rectangle, rounded corners, draw=black, below of = IDok, xshift=3cm, yshift=-0.5cm, fill=red!20,font=\scriptsize] (ID1) {JPG->PNG};
\node[rectangle, rounded corners, draw=black, below of = ID1, fill=red!20,font=\scriptsize] (ID2) {打包文件};

%lines
\draw [] (7.5, 2) -- (5,2);
\draw [->, >=stealth'] (5,2) -- (5,1);

\draw [->, >=stealth'] (ID1) -- (ID2);
\draw [->,>=stealth', very thick] (IDok) -- ($(IDok.east) + (4, 0)$);


\draw [->, >=stealth'] ($(IDok.east) + (4.2, -0.1)$) -- (ID1.north);

\draw (ID2.east) -- ($(ID2.east) + (3,0)$);
\draw [->, >=stealth'] ($(ID2.east) + (3,0)$) -- ($(ID2.east) + (3,1.8)$);

\end{tikzpicture}

\end{frame}

%  ######################
\begin{frame}[fragile]
\frametitle{定制开机动画}
  \framesubtitle{服务器端打包流程}

\begin{tikzpicture}
%\draw [rounded corners, draw=red, dashed] (0.6, 2.8) rectangle (9.7, -3.0);

\draw [rounded corners, fill=green!30,draw=green!30, dashed] (-2.2, 0.3) rectangle (2.6, -0.3);
\draw [rounded corners, fill=red!30,draw=red!30, dashed] (-2.2, -0.4) rectangle (2.6, -0.7);

%file table
\node [rectangle](IDtbl) {\bgroup
\def\arraystretch{1.15} 
\scriptsize
\begin{tabular}{|l|l|l|l|}
\hline
{cid} & {md5(key)} & {suffix} & {type} \\
\hline
{100} & {xx-xxxx} & {jpg} & {image} \\
{100} & {yy-yyyy} & {png} & {image} \\
{100} & {zz-zzzz} & {zip} & {boot} \\
\hline
\end{tabular}
\egroup
}; 
\node[rectangle,font=\scriptsize, above of = IDtbl, yshift=-0.02cm] {uc.makerfiles};

%server process steps
\node[rectangle, rounded corners, draw=black, right of = IDtbl, xshift=3.6cm, yshift=1.5cm, fill=red!20,font=\scriptsize] (ID1) {JPG->PNG};
\node[rectangle, rounded corners, draw=black, below of = ID1, fill=red!20,font=\scriptsize] (ID2) {打包文件};

%lines
\draw [] (2.5,0) -- (3.0,0);
\draw [] (3.0,0.0) -- (3.0, 1.5);
\draw [->, >=stealth'] (3.0, 1.5) -- (3.9, 1.5);

\draw [->, >=stealth'] (ID1) -- (ID2);


\draw [->] (ID2.east) -- ($(ID2.east) + (0.8,0)$);
%\draw [->, >=stealth'] ($(ID2.east) + (3,0)$) -- ($(ID2.east) + (3,1.8)$);

\node at (7.5, -0.5) [mybox, font=\tiny] () {\begin{lstlisting}[frame=none]
|-- desc.txt
|-- part0
|   `-- part0.png
|-- part1
|   `-- part1.png
...  ...
|-- part12
|   `-- part12.png
                              \end{lstlisting}
};

\node [mybox, fill=red!20, below of = ID2, yshift=-0.9cm, font=\tiny](ID3)  {\$ zip -r -Z store b.zip *};
\draw [->] ($(ID3.east) + (0.4,0)$) -- (ID3.east);

\node at (0, -2.3) [mybox, fill=red!20, font=\tiny](IDfullpackage) {定制包};
\draw [->] ($(IDfullpackage) + (0, 1.6cm)$) -- (IDfullpackage.north);

\draw [->] (ID3.west) .. controls (1.2, -2.0)  .. ($(ID3.west) + (-3cm, 0.4cm)$);

\node at (2, -3.3) [mybox, fill=black!20, font=\tiny](IDunpack) {解包};
\draw [->] ($(IDunpack.west) + (-1.6, 0)$) -- (IDunpack.west);
\draw [] ($(IDunpack.west) + (-1.6, 0)$) -- ($(IDunpack.west) + (-1.6, 0.6)$);

\node at (5, -3.3) [rectangle, font=\tiny](ID4) {/data/local/bootanimation.zip};
\draw [->] ($(ID4.south) + (-2.7cm, 0.2cm)$) .. controls (3.3, -3.9)  .. (ID4.south);
\draw [rounded corners, draw=red, dashed] (0.6, -2.8) rectangle (8.5, -4.0);
\end{tikzpicture}

\end{frame}


%  ######################
\begin{frame}
\frametitle{定制锁屏图像}
  \framesubtitle{服务器端处理}
  TODO
\begin{tikzpicture}
 
\end{tikzpicture}

\end{frame}

%  ######################
\begin{frame}
\frametitle{定制预装APP}
  \framesubtitle{服务器端处理}
  TODO
\begin{tikzpicture}
 
\end{tikzpicture}

\end{frame}


\subsection{Web接口设计}

%  ######################
\begin{frame}
\frametitle{Web接口设计与定义}

\begin{itemize}
 \item 交换格式为JSON
 \item 置于/v1/maker/的URL空间下
 \item 各个功能模块的请求接口:
 
 \vspace{0.2cm}
 %table
 \bgroup
\def\arraystretch{1.15} 
\scriptsize
\begin{tabular}{|c|l|l|}
\hline
{id} &{URL} & {说明}\\
\hline
{1} &{/v1/maker/begin} & {开始定制内容(第\ref{lblMakerBegin}页)}  \\
{2} &{/v1/maker/uptoken} & {定制内容上传凭证的获取(第\ref{lblMakreUptoken}页)}  \\
{3} & {/v1/maker/uploaded} & {通知服务器已经上传的素材(第\ref{lblMakerUploaded}页)} \\
{4} & {/v1/maker/gen} & {开始制作定制内容包(第\ref{lblMakreGen}页)}  \\
{5} & {/v1/maker/down} & {从服务器下载定制包(第\ref{lblMakerDown}页)} \\
{6} & {/v1/maker/finish} & {手机成功得到定制包(第\ref{lblMakerDown}页)} \\
{7} & {/v1/maker/fetch} & {获取当前用户所有已经制作的定制包信息(第\ref{}页)} \\
{8} & {/v1/maker/share} & {用户分享定制包(第\ref{}页)} \\
{9} & {/v1/maker/modify} & {用户修改定制包(第\ref{lblMakerModify}页)} \\
{10} & {/v1/maker/fetch} & {取当前用户所有定制包极其状态(第\ref{??}页)} \\
\hline
\end{tabular}
\egroup

\end{itemize}

\end{frame}

%  ######################
\begin{frame}[fragile]\label{lblMakerBegin}
\frametitle{Web接口设计与定义}
  \framesubtitle{开始定制的流程}

\begin{enumerate}
 \item 选择购机类型
 \item 添加本次定制的描述信息(可选)
 \item 开始制作自定义的操作
\end{enumerate}

上述过程需要接口入DB库:

\begin{tikzpicture}
\draw [rounded corners, draw=red!20,fill=red!20, dashed] (4.3, 1.3) rectangle (11.2, -2.1);
\node [mybox, font=\tiny] (ID1) {浏览器};
\draw (4.5,-1.6) rectangle (5.5, 1.0) node [pos=.5, font=\tiny] {服务器};

\node [right of = ID1, xshift=1.2cm, font=\tiny] () {URL: /v1/maker/begin};

%interface fields
\draw [->] ($(ID1.east) + (0cm, 0.3cm)$) -- ($(ID1.east) + (4cm, 0.3cm)$) node [above, pos=.5]{\bgroup
\def\arraystretch{1.15} 
\tiny
\begin{tabular}{|l|l|l|}
\hline
{cid} & {model} & {description} \\
\hline
{100} & {M2D} & {爸爸的特制}  \\
\hline
\end{tabular}
\egroup
}; 

\draw [->] ($(ID1.east) + (4cm, -0.3cm)$) -- ($(ID1.east) + (0cm, -0.3cm)$) node [below, pos=.5](IDresponse)
{\bgroup
\def\arraystretch{1.15} 
\tiny
\begin{tabular}{|l|l|}
\hline
{code} & {makerID} \\
\hline
{0} & {21} \\
\hline
\end{tabular}
\egroup
}; 

\node[rectangle,below of = IDresponse, xshift=-1.5cm, yshift=0.2cm, font=\tiny] (IDresCode) {0为成功,可参见第\ref{lblServerStatusCode}页的定义};

\node[rectangle,below of = IDresponse, xshift=-0.5cm, yshift=-0.2cm, font=\tiny] () {makerID唯一标示用户的定制包};


\node at (9.5, -0.45) [cylinder , aspect=0.25,  draw, minimum width=1.5cm,
                 cylinder body fill=blue!30, 
                 cylinder end fill=blue!20, shape border rotate=90,font=\tiny](IDdb){数据库};

\draw [->] ($(IDdb.west) + (-3.0cm, 0.4cm)$) -- ($(IDdb.west) + (0, 0.4cm)$) node [above,pos=.9,font=\tiny]{\bgroup
\def\arraystretch{1.15} 
\tiny
\begin{tabular}{|l|l|l|l|l|}
\hline
{id} & {cid} & {model} &{description} & {created}\\
\hline
{20} & {88} & {M3} & {特制-1} & {2015-1-1} \\
{21} & {100} & {M2D} & {爸爸的特制} & {2015-6-2} \\
\hline
\end{tabular}
\egroup
};

%draw mapping arrows
\draw [dashed, color=blue] (6.2,0.1) -- (6.2, -1.8);
\draw [dashed, color=blue] (6.2, -1.8) -- (4.0, -1.8);
\draw [dashed, color=blue] (4.0, -1.8) -- (4.0, -0.85);
\draw [->, >=stealth', color=blue] (4.0, -0.85) -- (3.5, -0.85);
\end{tikzpicture}

用户接下来所有定制内容数据(包括修改等),都是放在获取到的makerID范围下完成。

\end{frame}

%  ######################
\begin{frame}[fragile]\label{lblMakreUptoken}
\frametitle{Web接口设计与定义}
  \framesubtitle{取上传凭证}
复用当前网盘服务功能:

\vspace{0.1cm}

\begin{tikzpicture}
\draw [rounded corners, draw=red!20,fill=red!20, dashed] (7.3, 1.3) rectangle (9, -2.1);
\node [mybox, font=\tiny] (ID1) {浏览器};
\draw (7.5,-1.6) rectangle (8.5, 1.0) node [pos=.5, font=\tiny] {服务器};

%interface fields
\draw [->] ($(ID1.east) + (0cm, 0.3cm)$) -- ($(ID1.east) + (7cm, 0.3cm)$) node [above, pos=.5]{\bgroup
\def\arraystretch{1.15} 
\tiny
\begin{tabular}{|l|l|l|l|l|l|}
\hline
{username} & {token} & {filename} & {size} & {md5} & {collection}\\
\hline
{即CID} & {xxx} & {文件名} & {大小} & {MD5} & {}\\
\hline
\end{tabular}
\egroup
}; 

\draw [->] ($(ID1.east) + (7cm, -0.3cm)$) -- ($(ID1.east) + (0cm, -0.3cm)$) node [below, pos=.5](IDresponse)
{\bgroup
\def\arraystretch{1.15} 
\tiny
\begin{tabular}{|l|l|}
\hline
{code} & {token} \\
\hline
{0} & {xxxxxxx} \\
\hline
\end{tabular}
\egroup
}; 

\node [right of = ID1, xshift=2.2cm, font=\tiny] () {URL: /v1/netdisk/uploading};

\node[rectangle,below of = IDresponse, xshift=-1.5cm, yshift=0.2cm, font=\tiny] (IDresCode) {0为成功,可参见第\ref{lblServerStatusCode}页的定义};

\node[rectangle,below of = IDresponse, xshift=1.3cm, yshift=-0.2cm, font=\tiny] () {用户可以上传Qiniu服务器的凭证}; 
\end{tikzpicture}

测试脚本示例:
\begin{lstlisting}[basicstyle=\tiny,frame=box]
$ curl --header "Content-type:application/json"  --request POST
 -d '{"username":"300047","token":"592c597d-df55-4c34-bd75-54f09ee12441",
      "filename":"test.png", "size":"100", "md5":"c8883e85018e8c71dc860cf9b4608c00",
      "collection":"123"}'
 http://service.caredear.com/v1/netdisk/uploading 
\end{lstlisting}


\end{frame}

%  ######################
\begin{frame}[fragile]\label{lblMakerUploaded}
\frametitle{Web接口设计与定义}
  \framesubtitle{文件上传成功}

此时客户端(目前是浏览器)已经成功上传文件到Qiniu服务器,需要此接口通知定制服务器:

\begin{tikzpicture}
\draw [rounded corners, draw=red!20,fill=red!20, dashed] (6.8, 1.2) rectangle (11.8, -2.6);
\node [mybox, font=\tiny] (ID1) {浏览器};
\draw (7.5,-0.4) rectangle (8.5, 0.4) node [pos=.5, font=\tiny] (IDserver){服务器};

%interface fields
\draw [->] ($(ID1.east) + (0cm, 0.3cm)$) -- ($(ID1.east) + (7cm, 0.3cm)$) node [above, pos=.5]{\bgroup
\def\arraystretch{1.15} 
\tiny
\begin{tabular}{|l|l|l|l|l|}
\hline
{cid} & {makerID} & {filename} & {md5}  & {type}\\
\hline
{100} & {21} & {a.png} & {MD5sum}  & {boot-x}\\
\hline
\end{tabular}
\egroup
}; 

\draw [->] ($(ID1.east) + (7cm, -0.3cm)$) -- ($(ID1.east) + (0cm, -0.3cm)$) node [below, pos=.5](IDresponse)
{\bgroup
\def\arraystretch{1.15} 
\tiny
\begin{tabular}{|l|}
\hline
{code} \\
\hline
{0}  \\
\hline
\end{tabular}
\egroup
}; 

\node[rectangle,below of = IDresponse, xshift=-1.5cm, yshift=0.2cm, font=\tiny] (IDresCode) {0为成功,可参见第\ref{lblServerStatusCode}页的定义};

\node [right of = ID1, xshift=3.2cm, font=\tiny] () {URL: /v1/maker/uploaded};

\node [cylinder , below of = IDserver, yshift=-0.2cm, 
                  aspect=0.25,  draw, minimum width=1.5cm,
                 cylinder body fill=blue!30, 
                 cylinder end fill=blue!20, shape border rotate=90,font=\tiny](IDdb){数据库};

%\draw [->] ($(IDdb.west) + (-3.0cm, 0.4cm)$) -- ($(IDdb.west) + (0, 0.4cm)$) 
\node [below of = IDdb,xshift=1.5cm, yshift=0.1cm, font=\tiny]{\bgroup
\def\arraystretch{1.15} 
\tiny
\begin{tabular}{|l|l|l|l|}
\hline
{makerID}  & {md5} &{suffix} & {type}\\
\hline
{21}  & {MD5sum} & {png} & {boot-x} \\
\hline
\end{tabular}
\egroup
};

%arrows..
\draw [->] (IDserver.south)-- (IDdb);
\draw [->] (IDdb.south) -- ($(IDdb.south) + (0cm, -0.3cm)$);
\end{tikzpicture}

通知网盘:
\begin{tikzpicture}
\draw [-> ] (0,0) -- (4,0) node [above, pos=.5]()
{\bgroup
\def\arraystretch{1.15} 
\tiny
\begin{tabular}{|l|l|l|l|l|l|}
\hline
{username} & {token} & {filename} & {size} & {md5} & {collection}\\
\hline
{即CID} & {xxx} & {文件名} & {大小} & {MD5} & {}\\
\hline
\end{tabular}
\egroup
};

\node at (7,0) [rectangle, font=\tiny] {/v1/netdisk/uploaded};
\end{tikzpicture}

命令:
\begin{lstlisting}[basicstyle=\tiny,frame=box]
$ curl --header "Content-type:application/json"  --request POST
 -d '{"username":"300047","token":"592c597d-df55-4c34-bd75-54f09ee12441",
      "filename":"test.png", "size":"100", "md5":"c8883e85018e8c71dc860cf9b4608c00",
      "collection":"123"}'
 http://service.caredear.com/v1/netdisk/uploaded 
\end{lstlisting}

\end{frame}


%  ######################
\begin{frame}[fragile]\label{lblMakreGen}
\frametitle{Web接口设计与定义}
  \framesubtitle{确认制作定制包}
  

\begin{tikzpicture}
 \draw [rounded corners, draw=red!20,fill=red!20, dashed] (4.3, 1.3) rectangle (11.2, -3.9);
\node [mybox, font=\tiny] (ID1) {浏览器};
\draw (4.5,-0.6) rectangle (7.5, 0.6) node [pos=.5, font=\tiny] {服务器};

%interface fields
\draw [->] ($(ID1.east) + (0cm, 0.3cm)$) -- ($(ID1.east) + (4cm, 0.3cm)$) node [above, pos=.5]{\bgroup
\def\arraystretch{1.15} 
\tiny
\begin{tabular}{|l|l|}
\hline
{cid} & {makerID} \\
\hline
{100} & {21} \\
\hline
\end{tabular}
\egroup
}; 

\draw [->] ($(ID1.east) + (4cm, -0.3cm)$) -- ($(ID1.east) + (0cm, -0.3cm)$) node [below, pos=.5](IDresponse)
{\bgroup
\def\arraystretch{1.15} 
\tiny
\begin{tabular}{|l|l|}
\hline
{code} & {PIN码} \\
\hline
{0} & {AB-XX-XX} \\
\hline
\end{tabular}
\egroup
}; 

\node [right of = ID1, xshift=1.2cm, font=\tiny] () {URL: /v1/maker/gen};

\node[rectangle,below of = IDresponse, xshift=-1.5cm, yshift=0.2cm, font=\tiny] (IDresCode) {0为成功,可参见第\ref{lblServerStatusCode}页的定义};

\node[rectangle,below of = IDresponse, xshift=-0.5cm, yshift=-0.2cm, font=\tiny] () {输入该码可得到定制包};


\node at (7.5, -2.45) [cylinder , aspect=0.25,  draw, minimum width=1.5cm,
                 cylinder body fill=blue!30, 
                 cylinder end fill=blue!20, shape border rotate=90,font=\tiny](IDdb){数据库};

%\draw [->] ($(IDdb.west) + (-3.0cm, 0.4cm)$) -- ($(IDdb.west) + (0, 0.4cm)$)
\node [above of = IDdb,yshift=0.1cm, font=\tiny] (IDrawtable){\bgroup
\def\arraystretch{1.15} 
\tiny
\begin{tabular}{|l|l|l|}
\hline
{makerID} & {md5} & {type}\\
\hline
{21} & {yyy} & {boot-1} \\
{21} & {xxx} & {boot-2} \\
{21} & {xzz} & {铃声} \\
\hline
\end{tabular}
\egroup
};

\node [below of = IDdb,yshift=0.2cm, font=\tiny](IDpackage){\bgroup
\def\arraystretch{1.15} 
\tiny
\begin{tabular}{|l|l|l|l|l|}
\hline
{id} & {cid} & {model} &{md5} & {PIN码}\\
\hline
{21} & {100} & {M2D} & {\colorbox{green!30}{定制包}} &{\colorbox{green!30}{AB-XX-XX}} \\
\hline
\end{tabular}
\egroup
};

\draw [] (IDrawtable.east) -- ($(IDrawtable.east) + (0.6cm, 0cm)$);
\draw [->] ($(IDrawtable.east) + (0.6cm, 0cm)$) -- ($(IDrawtable.east) + (0.6cm, -1.5cm)$);

\draw [dashed] ($(IDpackage.south) + (1.5cm, 0cm)$) -- ($(IDpackage.south) + (1.5cm, -0.4cm)$);
\draw [dashed] ($(IDpackage.south) + (1.5cm, -0.4cm)$) -- ($(IDpackage.south) + (-4.0cm, -0.4cm)$);
\draw [->, dashed] ($(IDpackage.south) + (-4.0cm, -0.4cm)$) -- ($(IDpackage.south) + (-4.0cm, 2.5cm)$);
\end{tikzpicture}
 
%==============================
\end{frame}
\begin{frame}[fragile]\label{lblMakerDown}
\frametitle{Web接口设计与定义}
  \framesubtitle{下载定制包}
  

\begin{tikzpicture}
 \draw [rounded corners, draw=red!20,fill=red!20, dashed] (4.3, 1.3) rectangle (11.7, -4.1);
\node [mybox, font=\tiny] (ID1) {21KE手机};
\draw (4.5,-0.6) rectangle (7.5, 0.6) node [pos=.5, font=\tiny] {服务器};

%interface fields
\draw [->] ($(ID1.east) + (0cm, 0.3cm)$) -- ($(ID1.east) + (4cm, 0.3cm)$) node [above, pos=.5]{\bgroup
\def\arraystretch{1.15} 
\tiny
\begin{tabular}{|l|}
\hline
{PING码} \\
\hline
{AB-xx-xx} \\
\hline
\end{tabular}
\egroup
}; 

\draw [->] ($(ID1.east) + (4cm, -0.3cm)$) -- ($(ID1.east) + (0cm, -0.3cm)$) node [below, pos=.5](IDresponse)
{\bgroup
\def\arraystretch{1.15} 
\tiny
\begin{tabular}{|l|l|}
\hline
{code} & {downloadURL} \\
\hline
{0} & {xxx.qiniudn.com/xx} \\
\hline
\end{tabular}
\egroup
}; 

\node [right of = ID1, xshift=1.2cm, font=\tiny] () {URL: /v1/maker/down};

\node[rectangle,below of = IDresponse, xshift=-1.5cm, yshift=0.2cm, font=\tiny] (IDresCode) {0为成功,可参见第\ref{lblServerStatusCode}页的定义};

\node[rectangle,below of = IDresponse, xshift=-0.5cm, yshift=-0.2cm, font=\tiny] () {定制包下载地址};


\node at (9.0, -2.45) [cylinder , aspect=0.25,  draw, minimum width=1.5cm,
                 cylinder body fill=blue!30, 
                 cylinder end fill=blue!20, shape border rotate=90,font=\tiny](IDdb){数据库};

%\draw [->] ($(IDdb.west) + (-3.0cm, 0.4cm)$) -- ($(IDdb.west) + (0, 0.4cm)$) 
\node [above of = IDdb,xshift=-1.0cm,font=\tiny]{\bgroup
\def\arraystretch{1.15} 
\tiny
\begin{tabular}{|l|l|l|l|l|l|l|}
\hline
{id} & {cid} & {model}& {MD5} &{description} & {PIN码} &{dirty}\\
\hline
{11} & {100} & {M2c} & {xxxx} & {爸爸的M2C} & {\colorbox{green!30}{AB-XX-XX}}& {0} \\
{21} & {100} & {M2D} & {zzzz} & {爸爸的特制} & {CD-XX-XX} & {0}\\
\hline
\end{tabular}
\egroup
};

\draw [dashed] (4.4,-1.4) -- ( 3.6, -1.4);
\draw [->, dashed] (3.6, -1.4) -- (3.6, -1.0);

%\draw [->] ($(IDdb.west) + (-3.0cm, -1.9cm)$) -- ($(IDdb.west) + (0, -1.9cm)$)
\node [below of = IDdb, xshift=-1.0cm, yshift=0.1cm, font=\tiny](IDafterdb){\bgroup
\def\arraystretch{1.15} 
\tiny
\begin{tabular}{|l|l|l|l|l|l|l|}
\hline
{id} & {cid} & {model}& {MD5} &{description} & {PIN码} &{dirty}\\
\hline
{11} & {100} & {M2c} & {xxxx} & {爸爸的M2C} & {\colorbox{green!30}{AB-XX-XX}}& {\colorbox{green!30}{1}} \\
{21} & {100} & {M2D} & {zzzz} & {爸爸的特制} & {CD-XX-XX} & {0}\\
\hline
\end{tabular}
\egroup
};

\draw [->] ($(IDafterdb.west) + (-4.5cm, 0.3cm)$) -- ($(IDafterdb.west) + (0.1cm, 0.3cm)$) node [above, pos=.5] {\bgroup
\def\arraystretch{1.15} 
\tiny
\begin{tabular}{|l|}
\hline
{PIN码} \\
\hline
{AB-xx-xx} \\
\hline
\end{tabular}
\egroup};

\node at (2, -3.42) [rectangle, font=\tiny] {URL: /v1/maker/finish};

\draw [->] ($(IDafterdb.west) + (0.1cm, -0.4cm)$) -- ($(IDafterdb.west) + (-4.5cm, -0.4cm)$) node [below, pos=.5, font=\tiny] {\bgroup
\def\arraystretch{1.15} 
\tiny
\begin{tabular}{|l|}
\hline
{code} \\
\hline
{0} \\
\hline
\end{tabular}
\egroup};

\end{tikzpicture}
  
\end{frame}

%  ######################
\begin{frame}[fragile]\label{lblMakerModify}
\frametitle{Web接口设计与定义}
  \framesubtitle{fetch}
  
初始阶段暂时不实现。

\end{frame}
%  ######################
\begin{frame}[fragile]\label{lblMakerModify}
\frametitle{Web接口设计与定义}
  \framesubtitle{修改定制包}
  
初始阶段暂时不实现。

\end{frame}
git 
\subsection{定制包的设计}
%  ######################
\begin{frame}
\frametitle{定制包的格式定义}
所有的定制内容条款,如何组织成一个包,并且可以正确的解析出来。

\begin{itemize}
 \item 包和设备的对应关系
 \item 打包
 \item 解包(客户端)
 \item 签名
 \item CRC???
\end{itemize}

zip all above + manifest file

\end{frame}

\subsection{数据库的设计}
%  ######################
\begin{frame}

\frametitle{数据库设计}
草图:

\vspace{0.2cm}
\includegraphics[scale=0.44]{db.jpg}

\end{frame}



%  ######################
\begin{frame}

\frametitle{数据库设计}
用户从统一帐号的数据库引出。

\vspace{0.2cm}

\begin{tikzpicture}
\node [rectangle](IDmaintbl) {\bgroup
\def\arraystretch{1.15} 
\tiny
\begin{tabular}{|l|l|}
\hline
{id} & {user} \\
\hline
{100} & {foo}  \\
{101} & {abc}  \\
\hline
\end{tabular}
\egroup
}; 
\node[rectangle,font=\tiny, above of = IDmaintbl, yshift=-0.4cm] {uc.uc\_passport};
 
%main maker tbl
\node [rectangle,right of = IDmaintbl, xshift=3.63cm, yshift=-1.7cm](IDmakertbl) {\bgroup
\def\arraystretch{1.15} 
\tiny
\begin{tabular}{|l|l|l|l|l|l|l|l|l|l|}
\hline
{id} & {cid} & {model} & {description} & {MD5} &{PIN码} & {size} & {status} & {shareflag} & {created}\\
\hline
{11} & {100} & {M2C} & {爸爸的M2C} & {} & {AB-XX-XX} & {} & {finished} & {} & {2015-1-1}\\
{12} & {100} & {M3} & {妈妈的M3} & {} & {} & {} & {paused} & {} &{2012-1-1}\\
{13} & {101} & {M2D} & {my-1} & {} & {BD-XX-XX} & {} & {cancelled} & {} &{2014-06-09}\\
\hline
\end{tabular}
\egroup
}; 
\node[rectangle,font=\tiny, above of = IDmakertbl, yshift=-0.2cm] {uc.makerfiles};

\node [rectangle, below of = IDmakertbl, yshift=-1.6cm](IDmakerfile) {\bgroup
\def\arraystretch{1.15} 
\tiny
\begin{tabular}{|l|l|l|l|l|l|}
\hline
{makerID} & {filename} & {md5(key)} & {suffix} & {type} &{created} \\
\hline
{11} & {a.jpg} & {xx-xxxx} & {jpg} & {boot-1} & {}\\
{11} & {b.png} & {aa-xxxx} & {png} & {boot-2} & {} \\
{11} & {c.png} & {yy-yyyy} & {png} & {墙纸} & {}\\
{12} & {a.mp3} & {zz-zzzz} & {mp3} & {铃声} & {} \\
{12} & {b.jpg} & {cc-xxxx} & {jpg} & {boot-1} & {}\\
{12} & {a.jpg} & {xx-xxxx} & {jpg} & {boot-2} & {}\\
\hline
\end{tabular}
\egroup
}; 
\node[rectangle,font=\scriptsize, above of = IDmakerfile, yshift=0.2cm,font=\tiny] {uc.makerfiles};

\end{tikzpicture}

\end{frame}

%  ######################
\begin{frame}
 \frametitle{数据库中数据的设计}

packages表中的状态位(status):
\begin{itemize}
 \item 0 - 新建
 \item 1 - 制作中
 \item 99 - 完成
\end{itemize}


\end{frame}




%%%%%%%%%%%%%%%%%%%%%%%%%%%%%%%%%%%%%%%%%%%%%%%%%%%%%%%%%%%%%%
\section{小米帐号对接}

%  ######################
\begin{frame}
\frametitle{准备工作}
在dev.xiaomi.com网站注册成功如下企业开发者帐号:
\begin{itemize}
 \item 用户名:17705164171,密码:xiaomi123
 \item 企业名称: 深圳市小田科技有限公司南京分公司
\end{itemize}

在该帐号下创建一个应用,具体信息如下:
\begin{itemize}
 \item 包名:com.caredear.note
 \item AppID: 2882303761517332941
 \item AppKey: 5121733211941
 \item AppSecret: lFhFmxfRVDaW7PCXeQcYew==
\end{itemize}

在“帐号接入服务”菜单中,点击“启用”按钮,之后填入回调地址的URL

\end{frame}

%  ######################
\begin{frame}[fragile]
\frametitle{授权接口}

\begin{tikzpicture}
\node [mybox] (ID21k) {https://account.xiaomi.com/oauth2/authorize};

\end{tikzpicture}

尝试使用下面方法来调用:
\begin{lstlisting}
https://account.xiaomi.com/oauth2/authorize?client_id=2882303761517332941&redirect_uri=http%3A%2F%2Fservice2.caredear.com%2Fv1%2Fuc%2Fxiaomiaccount&response_type=token&userId=861457562

https://account.xiaomi.com/oauth2/authorize?client_id=2882303761517332941&redirect_uri=http://service2.caredear.com/v1/uc/xiaomiaccount&response_type=token
\end{lstlisting}


\end{frame}

%  ######################
\begin{frame}
\frametitle{APK接入}

示例代码在login/3rd\_party\_login/目录下

\end{frame}

%  ######################
\begin{frame}
\frametitle{调试}
  \framesubtitle{小米帐号对接}
  use openssl command and base64 command

\end{frame}

%%%%%%%%%%%%%%%%%%%%%%%%%%%%%%%%%%%%%%%%%%%%%%%%%%%%%%%%%%%%%%
\section{代码调用示例}
%  ######################
\begin{frame}[fragile]
\frametitle{Python与memcached}
  \framesubtitle{安装}

从http://ftp.tummy.com/pub/python-memcached/获取最新代码包,并安装:

\begin{lstlisting}[backgroundcolor=\color{light-gray}]
$ tar -zxvf python-memcached-1.54.tar.gz
$ python setup.py install
\end{lstlisting}

\vspace{0.3cm}

{\Large{说明:}}

代码的线程安全性 $ = \left\{ \begin{array}{l l}
                Python  & \ge 2.4  \\
                python-memcached & \ge 1.36 
               \end{array} \right.
$

\end{frame}

\begin{frame}[fragile]
\frametitle{Python与memcached}
  \framesubtitle{使用}

代码示例:

\begin{tikzpicture}
 \node [mybox](ID1) {\begin{lstlisting}[frame=none]
import memcache

mc = memcache.Client(['127.0.0.1:12000'],debug=0)

mc.set('keyname', 'key value')
mc.set('timedkey', 'value', 20)

value = mc.get('keyname')

mc.delete('timedkey')
         \end{lstlisting}
};

\node [rectangle, right of = ID1, yshift = 3cm, xshift=1.5cm, font=\tiny] (ID2){memcached服务器IP地址:端口号};

\draw [->,thick, red] ($(ID2.south) + (-0.6cm, -1.3cm)$) -- (ID2.south);

\node [rectangle, right of = ID1, yshift = -0.8cm, xshift=2.6cm, font=\tiny] (ID3) {key有效期20秒(默认是0,表示永远有效)};

\draw [->, thick, red] ($(ID3.north) + (-2cm, 0.4cm)$) .. controls ($(ID3.north) + (-0.3cm, 0.5cm)$) .. (ID3.north);

\node [rectangle, right of = ID1, yshift = -1.8cm, xshift=1.7cm, font=\tiny](IDdel) {删除一个key};
\draw [->, thick, red] ($(IDdel.west) + (-2.8cm, -0.4cm)$) .. controls ($(IDdel.west) + (-1.3cm, -0.5cm)$) .. (IDdel.west);

\end{tikzpicture}

  
\end{frame}


\end{document}
