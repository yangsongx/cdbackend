\documentclass{beamer}
\usetheme{Madrid}
%\usetheme{Hannover}
\usepackage{fontspec,xunicode,xltxtra,xecolor,listings,color}
\usepackage{bchart}
\usepackage{tikz}
\usetikzlibrary{arrows,shapes,trees,calc,automata,positioning,decorations}
\tikzset{
  mybox/.style={
    rectangle,
    rounded corners,
    draw=black
  },
}

\setmainfont{WenQuanYi Micro Hei}
\setsansfont{WenQuanYi Micro Hei}
\setmonofont{WenQuanYi Micro Hei}
\XeTeXlinebreaklocale "zh"
\XeTeXlinebreakskip=0pt plus 1pt minus 0.1pt

\begin{document}

\defverbatim[colored]\apicode{%
  \begin{lstlisting}[basicstyle=\tiny,language=java,frame=lines,showspaces=false,emphstyle=\color{blue}]
import com.caredear.common.util.CdRecycleUsrData;

int CdRecycleUsrData.saveDeletedContactData(String deleted_contact);
int CdRecycleUsrData.restoreDeletedContactData(Context context);
\end{lstlisting}
}

\title{亲连服务器开发设计文档}
\institute{CearDear Service Team}
\frame{\titlepage}

\begin{frame}
\frametitle{Outline}
\tableofcontents
\end{frame}

%%%%%%%%%%%%%%%%%%%%%%%%%%%%%%%%%%%%%%%%%%%%%%%%%%%%%%%%%%%%%%
\section{新用户注册设计}

%  ######################
\begin{frame}
\frametitle{申请Token}


\end{frame}

%%%%%%%%%%%%%%%%%%%%%%%%%%%%%%%%%%%%%%%%%%%%%%%%%%%%%%%%%%%%%%
\section{网盘设计}

%  ######################
\begin{frame}[fragile]
\frametitle{架构}

\begin{tikzpicture}
% the input from APK
\node [circle,draw,fill=black!20,thick,inner sep=0pt,minimum size=6mm] (IDApk) {APK};
% proto section
\node [mybox,right of = IDApk,xshift=5cm, yshift=3cm] (IDProto) {\begin{lstlisting}[frame=none,basicstyle=\tiny]
message  req{
  required int id = 0;
  required string name = 1;
}
message  ret{
  required int code = 0;
}
                     \end{lstlisting}
};
\node [rectangle, above of = IDProto,yshift=0.2cm, font=\tiny] () {backend/proto\_data/NetdiskMessage.proto};

\node [mybox, right of = IDApk, xshift = 3cm] (IDJava) {Java};
\node [mybox, right of = IDJava, xshift = 3cm] (IDCpp) {C++};

%below arraows
\draw [->,dashed, very thick] (IDProto) -- (IDJava);
\draw [->,dashed, very thick] (IDProto) -- (IDCpp);

\path (IDJava) edge[bend left=20] node[above] {setXXX()} (IDCpp);
\path (IDCpp) edge[bend left=20] node[below] {getXXX()} (IDJava);

\end{tikzpicture}
\end{frame}


%  ######################
\begin{frame}[fragile]
\frametitle{上传文件}

\begin{tikzpicture}
% the input from APK
\node [circle,draw,fill=black!20,thick,inner sep=0pt,minimum size=6mm] (IDApk) {APK};
\node [mybox,right of = IDApk,xshift=0.2cm,yshift=2cm] (ID1) {\begin{lstlisting}[frame=none,basicstyle=\tiny]
{
  "user":"130xxxxxxxx",
  "filename":"abc.png",
  "md5":xxxxxxx
}
                     \end{lstlisting}
};

\node [mybox, right of = IDApk, xshift=4cm,font=\small] (IDWeb) {Web前端};

\node [mybox, right of = IDWeb, xshift=4cm,yshift=2cm,font=\small] (IDTauth) {验证服务};
\node [rectangle,above of = IDTauth,yshift=-0.5cm] () {tauth};
\node [mybox, right of = IDWeb, xshift=4cm,yshift=-2cm,font=\small] (IDNetdisk) {网盘服务};

\node [circle,draw=blue!50,fill=blue!20,thick,inner sep=0pt,minimum size=6mm,below of = IDWeb,xshift=-2.5cm, yshift=-1.5cm,font=\small] (IDQiniu) {Qiniu服务器};
%below are arrow lines...
\draw [->, thick] (IDWeb) -- (IDApk);
%below are layer

\end{tikzpicture}
\end{frame}


%  ######################
\begin{frame}
\frametitle{下载文件}

\end{frame}


%  ######################
\begin{frame}
\frametitle{删除文件}

\begin{tikzpicture}
\node [circle,draw=blue!50,fill=blue!20,thick,inner sep=0pt,minimum size=6mm] (ID1) {Good};
\end{tikzpicture}

\end{frame}

%  ######################
\begin{frame}
\frametitle{用户网盘文件列举}

注意事项:当用户文件列表数据过大,多于1K时候,
是否需要通过zip等方式来完成?

\end{frame}

%  ######################
\begin{frame}
\frametitle{网盘数据库}

\begin{tikzpicture}
\node [circle,draw=blue!50,fill=blue!20,thick,inner sep=0pt,minimum size=6mm,font=\scriptsize] (IDdb) {netdisk};

\node [rectangle, below of = IDdb,yshift=-1.5cm] (ID1) {\scriptsize \begin{tabular}{|c|}
\hline
{USERS} \\
\hline
{FILES} \\
\hline
                      \end{tabular}};
% tabls..
\node [rectangle, right of = ID1,xshift=5.5cm, yshift=3cm] (IDUsers) {\tiny \begin{tabular}{|l|l|l|}
\hline
{USER\_NAME}  & {USED\_SIZE} & {USER\_QUOTA} \\
\hline
{130xxx} & {32} & {1000}  \\
{138xxx} & {99} & {1000}  \\
\hline
                      \end{tabular}};


\node [rectangle, right of = ID1,xshift=5.5cm, yshift=-3cm] (IDFiles) {\tiny \begin{tabular}{|l|l|l|l|}
\hline
{HASH\_KEY}  & {SIZE} & {FILENAME} & {OWNER}\\
\hline
{FILES}  & {} & {} & {}\\
\hline
                      \end{tabular}};
\end{tikzpicture}

\end{frame}

%%%%%%%%%%%%%%%%%%%%%%%%%%%%%%%%%%%%%%%%%%%%%%%%%%%%%%%%%%%%%%
\section{DB分布式集群及读写数据}

%  ######################
\begin{frame}
\frametitle{架构}


\end{frame}

%  ######################
\begin{frame}
\frametitle{使用示例}


\end{frame}
%%%%%%%%%%%%%%%%%%%%%%%%%%%%%%%%%%%%%%%%%%%%%%%%%%%%%%%%%%%%%%
\section{Debug and Tips}

%  ######################
\begin{frame}
\frametitle{调试}
  \framesubtitle{运行情况}
\begin{tikzpicture}
\node [mybox] (ID21k) {21k-code};
\node [mybox, right of = ID21k, xshift=2.8cm] (IDksadmin) {ksadmin};
\node [mybox, right of = IDksadmin, xshift=2.9cm, yshift=1.5cm] (IDksapp1) {ksapp1};
\node [mybox, right of = IDksadmin, xshift=2.9cm, yshift=-1.5cm] (IDksapp3) {ksapp3};
% arrow
\draw [->] (IDksadmin) -- (IDksapp1) node [pos=.4,above,font=\footnotesize,color=orange, sloped] {ksapp1};
\draw [->] (IDksadmin) -- (IDksapp3) node [pos=.4,above,font=\footnotesize,color=orange, sloped] {ksapp3};
\draw [->] (ID21k) -- (IDksadmin) node [pos=.5,above,font=\footnotesize,color=orange] {ksadmin};
\node [green!90, above] at (IDksapp1.north) {UserCenter,Web前端};
\node [green!90, above] at (IDksapp3.north) {Token验证服务器};
\end{tikzpicture}

服务端口分配:
\bgroup
\def\arraystretch{1.15} 
\begin{tabular}{|c|c|}
{服务名称} & {端口号}\\
        \hline
{注册Token} & {11999(暂定)}  \\
{Token验证(tauth)} & {12000}  \\
{网盘服务} & {12001} \\
\end{tabular}
\egroup
\end{frame}

%  ######################
\begin{frame}
\frametitle{调试}
  \framesubtitle{Log文件位置}
\begin{tikzpicture}
\node [mybox] (ID21k) {21k-code};
\node [mybox, right of = ID21k, xshift=1.0cm] (IDksadmin) {ksadmin};
\node [mybox, right of = IDksadmin, xshift=2.5cm, yshift=1.5cm] (IDksapp1) {ksapp1};
\node [mybox, right of = IDksadmin, xshift=2.5cm, yshift=-1.5cm] (IDksapp3) {ksapp3};
% arrow
\draw [->] (IDksadmin) -- (IDksapp1);
\draw [->] (IDksadmin) -- (IDksapp3);
\draw [->] (ID21k) -- (IDksadmin);
\node [green!90, above] at (IDksapp1.north) {UserCenter,Web前端};
\node [green!90, above] at (IDksapp3.north) {Token验证服务器};
\node [blue!90, right, font=\scriptsize] at (IDksapp1.east) {/opt/usercenter/logs/application.log};
\node [blue!90, right, font=\scriptsize] at (IDksapp3.east) {/home/tokenverify/nohup.out};
\end{tikzpicture}

\end{frame}

%  ######################
\begin{frame}
\frametitle{调试}
  \framesubtitle{交换文件命令}
\begin{tikzpicture}
\node [mybox] (ID21k) {21k-code};
\node [mybox, right of = ID21k, xshift=1.0cm] (IDksadmin) {ksadmin};
\node [mybox, right of = IDksadmin, xshift=2.5cm, yshift=1.5cm] (IDksapp1) {ksapp1};
\node [mybox, right of = IDksadmin, xshift=2.5cm, yshift=-1.5cm] (IDksapp3) {ksapp3};
% arrow
\draw [->, thick] (IDksadmin) -- (IDksapp1);
\draw [->] (IDksadmin) -- (IDksapp3);
\draw [->] (ID21k) -- (IDksadmin);
\node [green!90, above] at (IDksapp1.north) {UserCenter,Web前端};
\node [green!90, above] at (IDksapp3.north) {Token验证服务器};
\node [blue!90, right, font=\scriptsize] at (IDksapp1.east) {/opt/usercenter/logs/application.log};
\node [blue!90, right, font=\scriptsize] at (IDksapp3.east) {/home/tokenverify/nohup.out};
\end{tikzpicture}

\end{frame}
%  ######################
\begin{frame}
\frametitle{调试}
  \framesubtitle{运算Token}
  use openssl command and base64 command

\end{frame}


\end{document}
